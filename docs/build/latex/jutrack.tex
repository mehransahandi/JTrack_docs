%% Generated by Sphinx.
\def\sphinxdocclass{report}
\documentclass[letterpaper,10pt,english]{sphinxmanual}
\ifdefined\pdfpxdimen
   \let\sphinxpxdimen\pdfpxdimen\else\newdimen\sphinxpxdimen
\fi \sphinxpxdimen=.75bp\relax

\PassOptionsToPackage{warn}{textcomp}
\usepackage[utf8]{inputenc}
\ifdefined\DeclareUnicodeCharacter
% support both utf8 and utf8x syntaxes
  \ifdefined\DeclareUnicodeCharacterAsOptional
    \def\sphinxDUC#1{\DeclareUnicodeCharacter{"#1}}
  \else
    \let\sphinxDUC\DeclareUnicodeCharacter
  \fi
  \sphinxDUC{00A0}{\nobreakspace}
  \sphinxDUC{2500}{\sphinxunichar{2500}}
  \sphinxDUC{2502}{\sphinxunichar{2502}}
  \sphinxDUC{2514}{\sphinxunichar{2514}}
  \sphinxDUC{251C}{\sphinxunichar{251C}}
  \sphinxDUC{2572}{\textbackslash}
\fi
\usepackage{cmap}
\usepackage[T1]{fontenc}
\usepackage{amsmath,amssymb,amstext}
\usepackage{babel}



\usepackage{times}
\expandafter\ifx\csname T@LGR\endcsname\relax
\else
% LGR was declared as font encoding
  \substitutefont{LGR}{\rmdefault}{cmr}
  \substitutefont{LGR}{\sfdefault}{cmss}
  \substitutefont{LGR}{\ttdefault}{cmtt}
\fi
\expandafter\ifx\csname T@X2\endcsname\relax
  \expandafter\ifx\csname T@T2A\endcsname\relax
  \else
  % T2A was declared as font encoding
    \substitutefont{T2A}{\rmdefault}{cmr}
    \substitutefont{T2A}{\sfdefault}{cmss}
    \substitutefont{T2A}{\ttdefault}{cmtt}
  \fi
\else
% X2 was declared as font encoding
  \substitutefont{X2}{\rmdefault}{cmr}
  \substitutefont{X2}{\sfdefault}{cmss}
  \substitutefont{X2}{\ttdefault}{cmtt}
\fi


\usepackage[Bjarne]{fncychap}
\usepackage{sphinx}

\fvset{fontsize=\small}
\usepackage{geometry}


% Include hyperref last.
\usepackage{hyperref}
% Fix anchor placement for figures with captions.
\usepackage{hypcap}% it must be loaded after hyperref.
% Set up styles of URL: it should be placed after hyperref.
\urlstyle{same}

\usepackage{sphinxmessages}
\setcounter{tocdepth}{1}



\title{JuTrack}
\date{May 29, 2020}
\release{2.0}
\author{mehran Sahandi far}
\newcommand{\sphinxlogo}{\vbox{}}
\renewcommand{\releasename}{Release}
\makeindex
\begin{document}

\pagestyle{empty}
\sphinxmaketitle
\pagestyle{plain}
\sphinxtableofcontents
\pagestyle{normal}
\phantomsection\label{\detokenize{index::doc}}


JuTrack is a platform for remote assessments of an individual’s data in situ using smartphones.
JuTrack is more than a single application, you can create, manage a study and, share your collected data using one platform.

A wide range of measurements using motion sensors (e.g. acceleration, gyroscope) as well as social activities (e.g. time spent in social networks) are extracted from the device while it is being used. Recorded are activity types (e.g. walking, running), relative location, and ecological momentary assessments.
you may perform \sphinxstylestrong{Active} and \sphinxstylestrong{Passive} data collection and adopt the open\sphinxhyphen{}source codes for further requirements.

JuTrack platform includes:
\begin{itemize}
\item {} 
Android application

\item {} 
Python dashboard

\item {} 
python server API

\end{itemize}

\begin{DUlineblock}{0em}
\item[] 
\item[] 
\end{DUlineblock}

\noindent{\hspace*{\fill}\sphinxincludegraphics[scale=1.0]{{JuTrack}.png}\hspace*{\fill}}

\begin{DUlineblock}{0em}
\item[] 
\item[] 
\end{DUlineblock}


\chapter{JuTrack Social}
\label{\detokenize{JuTrack_Social:jutrack-social}}\label{\detokenize{JuTrack_Social::doc}}

\section{Requirements}
\label{\detokenize{JuTrack_Social:requirements}}\begin{itemize}
\item {} \begin{description}
\item[{JuTrack relay on \sphinxhref{https://play.google.com/store/apps/details?id=com.google.android.gms\&hl=en}{Google Play Service} .}] \leavevmode
If your Phone does not support it (e.g. some models of \sphinxstylestrong{Huawei} phones), you may not be able to use all modules.

\end{description}

\item {} 
You need to be in \sphinxstylestrong{Germany} play store, otherwise you may not be able to download the app. To change it \sphinxhref{https://support.google.com/googleplay/answer/7431675?hl=en}{visit}

\end{itemize}

\begin{DUlineblock}{0em}
\item[] 
\end{DUlineblock}


\section{How  to Install}
\label{\detokenize{JuTrack_Social:how-to-install}}
To install the  JuTrack Social application  go to the Google Play Store and search for \sphinxstylestrong{JuTrack Social},

\begin{DUlineblock}{0em}
\item[] Or click \sphinxhref{https://play.google.com/store/apps/details?id=inm7.Jutrack.jutrack\_Social}{here} to access JuTrack Social in Google Play Store.
\item[] 
\end{DUlineblock}

\noindent{\hspace*{\fill}\sphinxincludegraphics[scale=0.15]{{GooglePlayStore}.jpg}\hspace*{\fill}}

\begin{DUlineblock}{0em}
\item[] 
\end{DUlineblock}

download and Install the JuTrack Social, then open it, and follow the next steps:

\begin{DUlineblock}{0em}
\item[] 
\end{DUlineblock}


\section{How to Activate}
\label{\detokenize{JuTrack_Social:how-to-activate}}
First check your Internet connection, activation and joining a study require an Internet connection.
Then click on the \sphinxstylestrong{An Einer Studie Teilnehmen} (Join a study) on the welcome page.

\noindent{\hspace*{\fill}\sphinxincludegraphics[scale=0.15]{{JuTrackSocialStart}.jpg}\hspace*{\fill}}

\begin{DUlineblock}{0em}
\item[] 
\end{DUlineblock}

The QR\sphinxhyphen{}Code scanning page will be opened, to enable the scanning please accept camera permission.
Now, you can scan the provided Qr\sphinxhyphen{}Code within 1 minute.

\noindent{\hspace*{\fill}\sphinxincludegraphics[scale=0.15]{{JuTrackSocialCameraPermission}.jpg}\hspace*{\fill}}

\begin{DUlineblock}{0em}
\item[] 
\end{DUlineblock}

\begin{sphinxadmonition}{note}{Note:}
JuTrack Social will ask you to enable the Internet connection if it is not.
\end{sphinxadmonition}

\begin{sphinxadmonition}{important}{Important:}
provided QR\sphinxhyphen{}codes are for \sphinxstylestrong{One Time} usage, Using for a second time or in case of any problem, you will be redirected to \sphinxstylestrong{Retry} page.
\end{sphinxadmonition}

\begin{DUlineblock}{0em}
\item[] 
\end{DUlineblock}

In the next step, accept the necessary permissions. First, {\hyperref[\detokenize{JuTrack_Social:labellocationservice}]{\sphinxcrossref{\DUrole{std,std-ref}{Location Service}}}}  permission will pop\sphinxhyphen{}up, please enable it by selecting \sphinxstylestrong{ALLOW}

\begin{DUlineblock}{0em}
\item[] 
\end{DUlineblock}

\noindent{\hspace*{\fill}\sphinxincludegraphics[scale=0.15]{{JuTrackSocialLocationPermission}.jpg}\hspace*{\fill}}

\begin{DUlineblock}{0em}
\item[] 
\end{DUlineblock}

Then, JuTrack Social will take you to the \sphinxstylestrong{Data Usage Access} page in the Android setting to request for permission, which is necessary for {\hyperref[\detokenize{JuTrack_Social:labelapplicationusagestatistics}]{\sphinxcrossref{\DUrole{std,std-ref}{Application Usage Statistics}}}} module.

\begin{DUlineblock}{0em}
\item[] 
\end{DUlineblock}

\noindent{\hspace*{\fill}\sphinxincludegraphics[scale=0.15]{{JuTrackSocialUsageDataAccess}.jpg}\hspace*{\fill}}

\begin{DUlineblock}{0em}
\item[] 
\end{DUlineblock}

Just select JuTrack Social from a list and \sphinxstylestrong{enable} it.

\begin{DUlineblock}{0em}
\item[] 
\end{DUlineblock}

\noindent{\hspace*{\fill}\sphinxincludegraphics[scale=0.15]{{JuTrackSocialUsageDataAccessEnable}.jpg}\hspace*{\fill}}

\begin{DUlineblock}{0em}
\item[] 
\end{DUlineblock}

\sphinxstyleemphasis{Almost done!}, Just come back to the JuTrack Social by pressing back. Now depending on the Android version or Model of device, JuTrack may request to put it in  \sphinxstylestrong{White List}.
This step will request your Operation System (Android) or your device battery optimization program to permit JuTrack Social to work on background without of need for optimization.
This is a \sphinxstylestrong{vital} step to keep the JuTrack Social running smoothly.

\begin{DUlineblock}{0em}
\item[] 
\end{DUlineblock}

\noindent{\hspace*{\fill}\sphinxincludegraphics[scale=0.15]{{JuTrackSocialWhitelist}.jpg}\hspace*{\fill}}

\begin{DUlineblock}{0em}
\item[] 
\end{DUlineblock}

\begin{sphinxadmonition}{note}{Note:}
This step might be different for your device or version of Android.
\end{sphinxadmonition}

for \sphinxstylestrong{Samsung} devices with \sphinxstylestrong{Android 8 or above}:

\begin{DUlineblock}{0em}
\item[] 
\end{DUlineblock}

\noindent{\hspace*{\fill}\sphinxincludegraphics[scale=0.15]{{JuTrackSocialWhitelistSamsung}.jpg}\hspace*{\fill}}

\begin{DUlineblock}{0em}
\item[] 
\end{DUlineblock}

Open \sphinxstylestrong{unmounted apps} as follow:

\begin{DUlineblock}{0em}
\item[] 
\end{DUlineblock}

\noindent{\hspace*{\fill}\sphinxincludegraphics[scale=0.15]{{JuTrackSocialWhitelistSamsungEnabel}.jpg}\hspace*{\fill}}

\begin{DUlineblock}{0em}
\item[] 
\end{DUlineblock}

Then select JuTrack Social from a list and click done!

\begin{DUlineblock}{0em}
\item[] 
\end{DUlineblock}

\noindent{\hspace*{\fill}\sphinxincludegraphics[scale=0.15]{{JuTrackSocialWhitelistSamsungEnableSelected}.jpg}\hspace*{\fill}}

\begin{DUlineblock}{0em}
\item[] 
\end{DUlineblock}

For \sphinxstylestrong{Huawei} device:
select the JuTrack Social from a list and \sphinxstylestrong{disable} it.

\begin{DUlineblock}{0em}
\item[] 
\end{DUlineblock}

\noindent{\hspace*{\fill}\sphinxincludegraphics[scale=0.15]{{JuTrackSocialWhiteListHuawei}.jpg}\hspace*{\fill}}

\begin{DUlineblock}{0em}
\item[] 
\end{DUlineblock}

for \sphinxstylestrong{Xiaomi} devices :

\begin{DUlineblock}{0em}
\item[] 
\end{DUlineblock}

\noindent{\hspace*{\fill}\sphinxincludegraphics[scale=0.15]{{JuTrackSocialWhiteListXiaomi}.jpg}\hspace*{\fill}}

\begin{DUlineblock}{0em}
\item[] 
\end{DUlineblock}

select JuTrack from a list and \sphinxstylestrong{Enable} it.

\begin{DUlineblock}{0em}
\item[] 
\end{DUlineblock}

\sphinxstyleemphasis{Congratulation!}  you finished all the  necessary steps to Install and activate the JuTrack Social Application.
As you may realized, JuTrack Social will show a notification as long as its running on background!


\section{Main Menu}
\label{\detokenize{JuTrack_Social:main-menu}}
To access  the main menu of JuTrack click \sphinxstylestrong{three dots} on the right corner of main page.

\begin{DUlineblock}{0em}
\item[] 
\end{DUlineblock}

\noindent{\hspace*{\fill}\sphinxincludegraphics[scale=0.15]{{JuTrackSocialMainMenu}.jpg}\hspace*{\fill}}


\subsection{Administrator Page}
\label{\detokenize{JuTrack_Social:administrator-page}}
In this menu, you have access to some administrative option to control the behavior of JuTrack.
select \sphinxstylestrong{Einstellungen} from main menu, then insert the admin password (to get this password please contact us).

\begin{DUlineblock}{0em}
\item[] 
\end{DUlineblock}

\noindent{\hspace*{\fill}\sphinxincludegraphics[scale=0.28]{{JuTrackSocialAdminMenu}.png}\hspace*{\fill}}

\begin{sphinxadmonition}{note}{Note:}
you can enable the Phone data for synchronization (may cost extra charge depending on your carrier or contract).
\end{sphinxadmonition}

\begin{sphinxadmonition}{warning}{Warning:}
Most of the options will be assigned automatically by the server during activation. Recommended NOT TO CHANGE THEM.
\end{sphinxadmonition}

\begin{DUlineblock}{0em}
\item[] 
\end{DUlineblock}


\subsection{How to Leave a Study}
\label{\detokenize{JuTrack_Social:how-to-leave-a-study}}
you can leave a current study by selecting \sphinxstylestrong{Studie verlassen} (leave a study), you need to be connected to the Internet.
During the leaving process (after accepting it) all the collected data by JuTrack will be Synchronized with the Study server.

\begin{DUlineblock}{0em}
\item[] 
\end{DUlineblock}

\noindent{\hspace*{\fill}\sphinxincludegraphics[scale=0.15]{{JuTrackSocialMainMenu}.jpg}\hspace*{\fill}}

\begin{DUlineblock}{0em}
\item[] 
\end{DUlineblock}


\subsection{Manual Synchronization}
\label{\detokenize{JuTrack_Social:manual-synchronization}}
By selecting this option all data collected by JuTrack will be Synchronized with Study servers.

\begin{sphinxadmonition}{note}{Note:}
Only use when its necessary.
\end{sphinxadmonition}

\begin{DUlineblock}{0em}
\item[] 
\end{DUlineblock}


\section{Passive Monitoring:}
\label{\detokenize{JuTrack_Social:passive-monitoring}}
In passive monitoring modules, the assessment will continue without any interaction from users.
The following passive monitoring modules will collect data over specified criteria or time period automatically.

\begin{DUlineblock}{0em}
\item[] 
\end{DUlineblock}


\subsection{Location Service}
\label{\detokenize{JuTrack_Social:location-service}}\label{\detokenize{JuTrack_Social:labellocationservice}}
location service records the user’s relative location in a specified time period (i.e. every 10 minutes) or whenever there is a change is the user’s location.
for an energy\sphinxhyphen{}efficient implementation JuTrack benefits \sphinxhref{https://developer.android.com/training/location}{Google’s location service API}
also, to meet the requirements of security and privacy, the location transformed in 3D space randomly for each user.
this transformation is hidden for users and study owners which can not be reverted.

\begin{DUlineblock}{0em}
\item[] 
\end{DUlineblock}


\subsection{User Activity Detection}
\label{\detokenize{JuTrack_Social:user-activity-detection}}\label{\detokenize{JuTrack_Social:labeluseractivitydetection}}
It detects users mode of transportation, for an efficient implementation JuTrack benefits \sphinxhref{https://developers.google.com/android/reference/com/google/android/gms/location/ActivityRecognitionClient}{Google’s activity detection API}
in this module, JuTrack is able to detect up to 7 different activities every specified time (i.e.every 5 minutes ) or whenever a new activity detected.
the list of activities is as follow:
\begin{itemize}
\item {} 
IN\_VEHICLE

\item {} 
ON\_BICYCLE

\item {} 
ON\_FOOT

\item {} 
STILL

\item {} 
UNKNOWN

\item {} 
TILTING

\item {} 
WALKING

\item {} 
RUNNING

\end{itemize}

Besides detected activities, we also save an accuracy of detected activity in percentage.

\begin{sphinxadmonition}{note}{Note:}
we only record the detected activity with accuracy \textgreater{}75\%
\end{sphinxadmonition}

\begin{DUlineblock}{0em}
\item[] 
\end{DUlineblock}


\subsection{Application Usage Statistics}
\label{\detokenize{JuTrack_Social:application-usage-statistics}}\label{\detokenize{JuTrack_Social:labelapplicationusagestatistics}}
Application Usage Statistics keeps track of applications used by the user. It captures and records the amount of time each application was in the \sphinxstylestrong{foreground} since the \sphinxstylestrong{Previous Midnight}.
Among different applications such as social media, texting, games and etc, it also tracks the time spent on \sphinxstylestrong{phone calls} or Short Message Service (SMS) text messaging.
it records these statistics every specified time period (i.e. every 1 hour).

\begin{sphinxadmonition}{important}{Important:}
this module \sphinxstylestrong{does not} have access or track any personal information such as the content of the application or phone number.
\end{sphinxadmonition}

\begin{DUlineblock}{0em}
\item[] 
\end{DUlineblock}


\subsection{Sensors}
\label{\detokenize{JuTrack_Social:sensors}}
supported sensors:
\begin{itemize}
\item {} 
Accelerometer

\item {} 
Gyroscope

\item {} 
Barometer

\item {} 
Magnetometer

\item {} 
Gravity

\item {} 
Light

\item {} 
Linear Accelerometer

\item {} 
Rotation

\item {} 
Proximity

\end{itemize}

\begin{DUlineblock}{0em}
\item[] 
\end{DUlineblock}


\section{Active Monitoring}
\label{\detokenize{JuTrack_Social:active-monitoring}}
\sphinxstyleemphasis{will soon be released}


\section{Ecological Momentary Assessment}
\label{\detokenize{JuTrack_Social:ecological-momentary-assessment}}
\sphinxstyleemphasis{will soon be released}


\chapter{JuSERVER}
\label{\detokenize{JuServer:juserver}}\label{\detokenize{JuServer::doc}}

\section{API}
\label{\detokenize{JuServer:api}}

\section{Dashboard}
\label{\detokenize{JuServer:dashboard}}

\chapter{Developers}
\label{\detokenize{developers:developers}}\label{\detokenize{developers::doc}}

\section{SAVE TO DATABASE}
\label{\detokenize{developers:save-to-database}}

\section{Synchronization with server}
\label{\detokenize{developers:synchronization-with-server}}

\section{foreground services}
\label{\detokenize{developers:foreground-services}}

\section{crashHandler}
\label{\detokenize{developers:crashhandler}}

\section{OnBoarding}
\label{\detokenize{developers:onboarding}}

\section{server Interface}
\label{\detokenize{developers:server-interface}}

\chapter{how to contribute:}
\label{\detokenize{developers:how-to-contribute}}

\chapter{Source Code:}
\label{\detokenize{developers:source-code}}

\chapter{How to Cite :}
\label{\detokenize{index:how-to-cite}}

\chapter{Contact Us:}
\label{\detokenize{index:contact-us}}
Interested in using JuTrack in you study? contact us:

\begin{DUlineblock}{0em}
\item[] 
\end{DUlineblock}

To contribution as a development please check {\hyperref[\detokenize{developers::doc}]{\sphinxcrossref{\DUrole{doc}{Developers}}}} page.



\renewcommand{\indexname}{Index}
\printindex
\end{document}